%% abtex2-modelo-trabalho-academico.tex, v-1.9.6 laurocesar
%% Copyright 2012-2016 by abnTeX2 group at http://www.abntex.net.br/ 
%%
%% This work may be distributed and/or modified under the
%% conditions of the LaTeX Project Public License, either version 1.3
%% of this license or (at your option) any later version.
%% The latest version of this license is in
%%   http://www.latex-project.org/lppl.txt
%% and version 1.3 or later is part of all distributions of LaTeX
%% version 2005/12/01 or later.
%%
%% This work has the LPPL maintenance status `maintained'.
%% 
%% The Current Maintainer of this work is the abnTeX2 team, led
%% by Lauro César Araujo. Further information are available on 
%% http://www.abntex.net.br/
%%
%% This work consists of the files abntex2-modelo-trabalho-academico.tex,
%% abntex2-modelo-include-comandos and abntex2-modelo-references.bib
%%
% -------------------------------------------------------------------------
% -------------------------------------------------------------------------
%  abnTeX2: Modelo de Trabalho Academico (tese de doutorado, dissertacao de
%  mestrado e trabalhos monograficos em geral) em conformidade com 
%  ABNT NBR 14724:2011: Informacao e documentacao - Trabalhos academicos -
%  Apresentacao
% -------------------------------------------------------------------------
% -------------------------------------------------------------------------

\documentclass[
	% -- opções da classe memoir --
	12pt,				% tamanho da fonte
	openright,			% capítulos começam em pág ímpar (insere página vazia caso preciso)
	%twoside,			% para impressão em recto e verso. Oposto a oneside
	oneside,
	a4paper,			% tamanho do papel. 
	% -- opções da classe abntex2 --
	%chapter=TITLE,		% títulos de capítulos convertidos em letras maiúsculas
	%section=TITLE,		% títulos de seções convertidos em letras maiúsculas
	%subsection=TITLE,	% títulos de subseções convertidos em letras maiúsculas
	%subsubsection=TITLE,% títulos de subsubseções convertidos em letras maiúsculas
	% -- opções do pacote babel --
	english,			% idioma adicional para hifenização
	french,				% idioma adicional para hifenização
	spanish,			% idioma adicional para hifenização
	brazil				% o último idioma é o principal do documento
	]{abntex2}

% ---
% Pacotes básicos 
% ---
\usepackage{lmodern}			% Usa a fonte Latin Modern			
\usepackage[T1]{fontenc}		% Selecao de codigos de fonte.
\usepackage[utf8]{inputenc}		% Codificacao do documento (conversão automática dos acentos)
\usepackage{lastpage}			% Usado pela Ficha catalográfica
\usepackage{indentfirst}		% Indenta o primeiro parágrafo de cada seção.
\usepackage{color}				% Controle das cores
\usepackage{graphicx}			% Inclusão de gráficos
\usepackage{microtype} 			% para melhorias de justificação
\usepackage{multirow}
% ---
		
% ---
% Pacotes adicionais, usados apenas no âmbito do Modelo Canônico do abnteX2
% ---
\usepackage{lipsum}				% para geração de dummy text
% ---

% ---
% Pacotes de citações
% ---
\usepackage[brazilian,hyperpageref]{backref}	 % Paginas com as citações na bibl
\usepackage[alf]{abntex2cite}	% Citações padrão ABNT

% --- 
% CONFIGURAÇÕES DE PACOTES
% --- 

% ---
% Configurações do pacote backref
% Usado sem a opção hyperpageref de backref
\renewcommand{\backrefpagesname}{Citado na(s) página(s):~}
% Texto padrão antes do número das páginas
\renewcommand{\backref}{}
% Define os textos da citação
\renewcommand*{\backrefalt}[4]{
	\ifcase #1 %
		Nenhuma citação no texto.%
	\or
		Citado na página #2.%
	\else
		Citado #1 vezes nas páginas #2.%
	\fi}%
% ---

% ---
% Informações de dados para CAPA e FOLHA DE ROSTO
% ---
\titulo{Seleção dinâmica de interface em redes \\ IEEE 802.15.4}
\autor{Bruno Monteiro Pires}
\local{Rio Claro}
\data{2016}
\orientador{Alexandro José Baldassin}
\coorientador{Alex Roschildt Pinto}
\instituicao{
  Universidade Estadual Paulista -- UNESP
  \par
  Instituto de Biociências, Letras e Ciências Exatas
  \par
  Programa de Pós-Graduação em Ciência da Computação}
\tipotrabalho{Dissertação (Mestrado)}
% O preambulo deve conter o tipo do trabalho, o objetivo, 
% o nome da instituição e a área de concentração 

\preambulo{Dissertação de Mestrado elaborada junto ao Programa
	de Pós-Graduação em Ciência da Computação – Área
	de Concentração em Arquitetura de Computadores e
	Sistemas Distribuídos, como parte dos requisitos para
	obtenção do título de Mestre em Ciência da Computação.
}
% ---


% ---
% Configurações de aparência do PDF final

% alterando o aspecto da cor azul
\definecolor{blue}{RGB}{41,5,195}

% informações do PDF
\makeatletter
\hypersetup{
     	%pagebackref=true,
		pdftitle={\@title}, 
		pdfauthor={\@author},
    	pdfsubject={\imprimirpreambulo},
	    pdfcreator={LaTeX with abnTeX2},
		pdfkeywords={abnt}{latex}{abntex}{abntex2}{trabalho acadêmico}, 
		colorlinks=true,       		% false: boxed links; true: colored links
    	linkcolor=blue,          	% color of internal links
    	citecolor=blue,        		% color of links to bibliography
    	filecolor=magenta,      	% color of file links
		urlcolor=blue,
		bookmarksdepth=4
}
\makeatother
% --- 

% --- 
% Espaçamentos entre linhas e parágrafos 
% --- 

% O tamanho do parágrafo é dado por:
\setlength{\parindent}{1.3cm}

% Controle do espaçamento entre um parágrafo e outro:
\setlength{\parskip}{0.2cm}  % tente também \onelineskip

% ---
% compila o indice
% ---
\makeindex
% ---

% ---
% Set graphics path
% ---
\graphicspath{{images/}}

% ----
% Início do documento
% ----
\begin{document}

% Seleciona o idioma do documento (conforme pacotes do babel)
%\selectlanguage{english}
\selectlanguage{brazil}

% Retira espaço extra obsoleto entre as frases.
\frenchspacing 

% ----------------------------------------------------------
% ELEMENTOS PRÉ-TEXTUAIS
% ----------------------------------------------------------
% \pretextual

% ---
% Capa
% ---
\imprimircapa
% ---

% ---
% Folha de rosto
% (o * indica que haverá a ficha bibliográfica)
% ---
\imprimirfolhaderosto*
% ---

% ---
% Inserir a ficha bibliografica
% ---

% Isto é um exemplo de Ficha Catalográfica, ou ``Dados internacionais de
% catalogação-na-publicação''. Você pode utilizar este modelo como referência. 
% Porém, provavelmente a biblioteca da sua universidade lhe fornecerá um PDF
% com a ficha catalográfica definitiva após a defesa do trabalho. Quando estiver
% com o documento, salve-o como PDF no diretório do seu projeto e substitua todo
% o conteúdo de implementação deste arquivo pelo comando abaixo:
%
% \begin{fichacatalografica}
%     \includepdf{fig_ficha_catalografica.pdf}
% \end{fichacatalografica}

\begin{fichacatalografica}
	\sffamily
	\vspace*{\fill}					% Posição vertical
	\begin{center}					% Minipage Centralizado
	\fbox{\begin{minipage}[c][8cm]{13.5cm}		% Largura
	\small
	\imprimirautor
	%Sobrenome, Nome do autor
	
	\hspace{0.5cm} \imprimirtitulo  / \imprimirautor. --
	\imprimirlocal, \imprimirdata-
	
	\hspace{0.5cm} \pageref{LastPage} p. : il. (algumas color.) ; 30 cm.\\
	
	\hspace{0.5cm} \imprimirorientadorRotulo~\imprimirorientador\\
	
	\hspace{0.5cm}
	\parbox[t]{\textwidth}{\imprimirtipotrabalho~--~\imprimirinstituicao,
	\imprimirdata.}\\
	
	\hspace{0.5cm}
		1. IEEE802.15.4.
		I. Alexandro José Baldassin.
		II. Universidade Estadual Paulista.
		III. Instituto de Biociências, Letras e Ciências Exatas.
		IV. Seleção dinâmica de interface em redes IEEE 802.15.4
	\end{minipage}}
	\end{center}
\end{fichacatalografica}
% ---

% ---
% Inserir folha de aprovação
% ---

% Isto é um exemplo de Folha de aprovação, elemento obrigatório da NBR
% 14724/2011 (seção 4.2.1.3). Você pode utilizar este modelo até a aprovação
% do trabalho. Após isso, substitua todo o conteúdo deste arquivo por uma
% imagem da página assinada pela banca com o comando abaixo:
%
% \includepdf{folhadeaprovacao_final.pdf}
%
\begin{folhadeaprovacao}

  \begin{center}
    {\ABNTEXchapterfont\large\imprimirautor}

    \vspace*{\fill}\vspace*{\fill}
    \begin{center}
      \ABNTEXchapterfont\bfseries\Large\imprimirtitulo
    \end{center}
    \vspace*{\fill}
    
    \hspace{.45\textwidth}
    \begin{minipage}{.5\textwidth}
        \imprimirpreambulo
    \end{minipage}%
    \vspace*{\fill}
   \end{center}
        
   Trabalho aprovado. \imprimirlocal, 22 de dezembro de 2016:

   \assinatura{\textbf{\imprimirorientador} \\ Orientador} 
   \assinatura{\textbf{Professor} \\ Convidado 1}
   \assinatura{\textbf{Professor} \\ Convidado 2}
   %\assinatura{\textbf{Professor} \\ Convidado 3}
   %\assinatura{\textbf{Professor} \\ Convidado 4}
      
   \begin{center}
    \vspace*{0.5cm}
    {\large\imprimirlocal}
    \par
    {\large\imprimirdata}
    \vspace*{1cm}
  \end{center}
  
\end{folhadeaprovacao}
% ---

% ---
% Dedicatória
% ---
\begin{dedicatoria}
   \vspace*{\fill}
   \centering
   \noindent
   \textit{ Este trabalho é dedicado às crianças adultas que,\\
   quando pequenas, sonharam em se tornar cientistas.} \vspace*{\fill}
\end{dedicatoria}
% ---

% ---
% Agradecimentos
% ---
\begin{agradecimentos}
Os agradecimentos principais são direcionados à Gerald Weber, Miguel Frasson,
Leslie H. Watter, Bruno Parente Lima, Flávio de Vasconcellos Corrêa, Otavio Real
Salvador, Renato Machnievscz\footnote{Os nomes dos integrantes do primeiro
projeto abn\TeX\ foram extraídos de
\url{http://codigolivre.org.br/projects/abntex/}} e todos aqueles que
contribuíram para que a produção de trabalhos acadêmicos conforme
as normas ABNT com \LaTeX\ fosse possível.

Agradecimentos especiais são direcionados ao Centro de Pesquisa em Arquitetura
da Informação\footnote{\url{http://www.cpai.unb.br/}} da Universidade de
Brasília (CPAI), ao grupo de usuários
\emph{latex-br}\footnote{\url{http://groups.google.com/group/latex-br}} e aos
novos voluntários do grupo
\emph{\abnTeX}\footnote{\url{http://groups.google.com/group/abntex2} e
\url{http://www.abntex.net.br/}}~que contribuíram e que ainda
contribuirão para a evolução do \abnTeX.

\end{agradecimentos}
% ---

% ---
% RESUMOS
% ---

% resumo em português
%\setlength{\absparsep}{18pt} % ajusta o espaçamento dos parágrafos do resumo
%\begin{resumo}
% Segundo a \citeonline[3.1-3.2]{NBR6028:2003}, o resumo deve ressaltar o
% objetivo, o método, os resultados e as conclusões do documento. A ordem e a extensão
% destes itens dependem do tipo de resumo (informativo ou indicativo) e do
% tratamento que cada item recebe no documento original. O resumo deve ser
% precedido da referência do documento, com exceção do resumo inserido no
% próprio documento. (\ldots) As palavras-chave devem figurar logo abaixo do
% resumo, antecedidas da expressão Palavras-chave:, separadas entre si por
% ponto e finalizadas também por ponto.

% \textbf{Palavras-chave}: latex. abntex. editoração de texto.
%\end{resumo}

% resumo em inglês
%\begin{resumo}[Abstract]
% \begin{otherlanguage*}{english}
%   This is the english abstract.

%   \vspace{\onelineskip}
 
%   \noindent 
%   \textbf{Keywords}: latex. abntex. text editoration.
% \end{otherlanguage*}
%\end{resumo}
% ---

% ---
% inserir lista de ilustrações
% ---
\pdfbookmark[0]{\listfigurename}{lof}
\listoffigures*
\cleardoublepage
% ---

% ---
% inserir lista de tabelas
% ---
\pdfbookmark[0]{\listtablename}{lot}
\listoftables*
\cleardoublepage
% ---

% ---
% inserir lista de abreviaturas e siglas
% ---
%\begin{siglas}
%  \item[ABNT] Associação Brasileira de Normas Técnicas
%  \item[abnTeX] ABsurdas Normas para TeX
%\end{siglas}
% ---

% ---
% inserir lista de símbolos
% ---
%\begin{simbolos}
%  \item[$ \Gamma $] Letra grega Gama
%  \item[$ \Lambda $] Lambda
%  \item[$ \zeta $] Letra grega minúscula zeta
%  \item[$ \in $] Pertence
%\end{simbolos}
% ---

% ---
% inserir o sumario
% ---
\pdfbookmark[0]{\contentsname}{toc}
\tableofcontents*
\cleardoublepage
% ---



% ----------------------------------------------------------
% ELEMENTOS TEXTUAIS
% ----------------------------------------------------------
\textual

% ----------------------------------------------------------
% Introdução
% ----------------------------------------------------------
\chapter[Introdução]{Introdução}
Redes de sensores sem fios têm, desde seu surgimento, possibilitado o monitoramento dos mais diversos fenômenos e processos, além de abrir caminho para a criação de uma nova geração de sistemas inteligentes, dinâmicos e conectados. Historicamente esforços visionários como Igloo White \cite{Warneke2001} e Smart Dust \cite{Correl2004} demonstraram o potencial desta tecnologia. No entanto, até o momento, são raras as iniciativas realizadas no sentido de idealizar plataformas de uso geral.

A diversidade das aplicações as quais se destinam as redes de sensores sem fios impõe restrições no projeto e na implantação destes sistemas, o que favorece iniciativas de desenvolvimento orientadas à aplicação. Embora esta abordagem leve à criação de soluções eficazes para aplicações específicas, estabelece um padrão de desenvolvimento restrito e desconsidera possibilidades de otimização mais abrangentes. Além de inibir eventuais reduções de custo resultantes da produção em massa e difusão da tecnologia \cite{Rawat2014}. 

É possível supor que dentre as motivações para adoção generalizada deste modelo de desenvolvimento esteja o fato de as pesquisas na área terem se iniciado antes que as tecnologias que as viabilizam estivessem totalmente desenvolvidas. Tal consideração fica evidente quando se observa, por exemplo, a história de Igloo White. Uma operação militar responsável por monitorar os campos de batalha durante a guerra do Vietnã, através do que talvez tenha sido a primeira utilização documentada de uma rede de sensores sem fios. Na época foi necessário que cada um de seus componentes fosse projetado sob medida, assim como a infraestrutura necessária para coleta e posterior processamento dos dados obtidos. Isso fez com que a operação demandasse uma quantidade exorbitante de recursos, sendo sua execução assegurada somente após substanciais investimentos do departamento de defesa dos Estados Unidos.

Efetivamente diversas das motivações para que se desenvolvessem redes de sensores sem fios e sistemas de informação similares já existiam muito antes que as tecnologias disponíveis possibilitassem a implementação de soluções viáveis. Por este motivo a área sempre foi muito beneficiada pelos avanços realizados no campo da microeletrônica de baixa potência, que vêm acontecendo de forma regular. Isso faz com que frequentemente barreiras técnicas sejam superadas, alterando paradigmas de projeto e abrindo caminho para exploração de novos mercados e aplicações. Razão pela qual o tema continua recebendo tanta atenção, apesar de as pesquisas na área terem começado a se intensificar já no início dos anos 2000 \cite{Kahn2000, Pottie2000, Heinzelman2000}, quase duas décadas atrás.

Contudo o maior motivador para o crescente interesse no desenvolvimento de redes de sensores sem fios foi, sem dúvida, o surgimento do conceito de Internet das Coisas, ou \textit{Internet of Things} (IoT). Um paradigma tecnológico no qual se propõe que os mais diversos objetos devam ser equipados de forma a permitir que se comuniquem, diretamente e através da Internet, e que possam ser acessados remotamente a fim de prover informações úteis, ou ainda, realizar ações. Segundo \cite{Harrop2014}, os mercados relacionados às redes de sensores sem fios devem chegar a movimentar cerca de 1.8 bilhões de dólares até 2024, sendo esta soma referente somente aos sistemas de rede com topologia \textit{mesh}. Estima-se também um crescimento exponencial na quantidade de dispositivos conectados, podendo esta atingir um total de 25 bilhões de dispositivos até 2020 \cite{Want2015}. Para que esta demanda possa ser atendida, no entanto, ainda existem diversos desafios a serem superados. Se destacam questões como padronização dos protocolos de comunicação e interoperabilidade \cite{Rawat2014, IEC2014}, dentre outros tópicos vinculados à massificação desta tecnologia. 

Neste âmbito a obtenção de links de comunicação versáteis e confiáveis é essencial, sendo que o método mais disseminado atualmente é a transmissão de dados via radiofrequência. Se comparada a outras técnicas, a utilização de radiotransmissores oferece maior flexibilidade quanto à topologia das redes e simplifica o processo de implantação, pois permite que se integrem novos dispositivos à infraestrutura já existente sem que sejam necessárias grandes intervenções.

No entanto a qualidade dos links de rádio sofre influencia de diversas variáveis externas como, por exemplo: interferência gerada por outros equipamentos, alterações na vegetação, objetos se movendo nas proximidades e até mesmo mudanças nas condições climáticas. Logo a eficiência destes links tende a variar ao longo do tempo e de acordo com a localização dos transmissores \cite{Kusy2011}. Protocolos de roteamento modernos são capazes de lidar com estas adversidades detectando alterações na qualidade dos links individuais e estabelecendo rotas alternativas a fim de otimizar o fluxo de dados na rede e garantir maior eficiência energética \cite{Gnawali2009}. No entanto esta abordagem depende do provisionamento de rotas alternativas, requerendo redes mais densas, o que eleva os custos e pode dificultar o processo de implantação.

\section{Objetivos e Contribuições}
Este trabalho tem como objetivo principal explorar alternativas para mitigar os problemas de instabilidade relacionados aos links de rádio nas redes de sensores sem fios. Diversas iniciativas já foram realizadas com este intuito, contudo se trata de um espaço de soluções bastante amplo e diversificado, havendo ainda grande potencial para otimização. A abordagem proposta nas seções seguintes buscará utilizar mais intensivamente o hardware e as camadas inferiores dos protocolos de comunicação, a fim de permitir a implementação de plataformas mais flexíveis e robustas, capazes de operar em uma maior gama de situações. 

A fim de validar estas propostas experimentalmente será introduzida, na seção \ref{phynode}, a plataforma PhyNode. A abordagem utilizada em sua construção tem o potencial de contribuir para o estado da arte das pesquisas na área, pois complementa outras soluções já presentes na literatura \cite{Pantazis2013, Tarique2009} e que, em grande parte, são destinadas exclusivamente a ambientes de rede \textit{mesh} densamente populados e redundantes, além de não dedicarem atenção especial às possibilidades proporcionadas pela utilização de hardware especificamente destinado à manutenção de links de comunicação mais robustos.

É interessante que soluções como PhyNode sejam desenvolvidas com o objetivo de incentivar o processo de massificação do uso de sistemas baseados em redes de sensores sem fios. A medida que esta tecnologia é integrada a um número maior de produtos e disponibilizada aos consumidores finais, se torna essencial que seja capaz de operar de forma confiável em diferentes ambientes e através de topologias bastante diversificadas. Atualmente a maior parte das redes de sensores sem fios é utilizada em pesquisas científicas, com fins exploratórios, permitindo que cada projeto seja realizado considerando ambientes de aplicação específicos. Este paradigma, no entanto, não é adequado para as aplicações comerciais, onde a produção em massa se beneficia de projetos mais flexíveis, que possibilitem a produção em larga escala e sejam capazes de adaptar-se a diferentes aplicações.

\section{Organização do texto}
% TODO - Organização do texto
% ---

% ---
% Fundamentação teórica e trabalhos relacionados
% ---
\chapter{Fundamentação teórica}
A fim de embasar as discussões apresentadas ao longo deste trabalho, esta seção será dedicada a apresentar os diversos conceitos que serão futuramente utilizados, começando pela própria arquitetura associada às redes de sensores sem fios. Estes sistemas são definidos como redes compostas por um conjunto de nodos de sensores (também conhecidos como \textit{motes}), capazes de estabelecer uma infraestrutura de comunicação de forma \textit{ad-hoc}, e utiliza-la para coletar e transmitir dados obtidos a partir dos sensores presentes em cada nodo. Podem contar ainda com um ou mais \textit{sinks}, responsáveis por concentrar a coleta dos dados obtidos, e \textit{gateways}, responsáveis por fazer a ponte entre os nodos que compõem a rede e o exterior, disponibilizando os dados coletados a uma rede local ou transmitindo-os diretamente para a internet. A arquitetura ilustrada na \autoref{fig_node_generic} será considerada daqui em diante como o modelo básico de um nodo de sensores sem fio. 

\begin{figure}[htb]
	\caption{\label{fig_node_generic}Nodo de sensores sem fio: arquitetura básica}
	\begin{center}
		\includegraphics[width=\linewidth]{SensorNode_Generic}
	\end{center}
	\legend{Adaptado de \citeonline{Anastasi2009}}
\end{figure}

Cada nodo de sensor representado por esta arquitetura é composto essencialmente por quatro subsistemas \cite{Akyildiz2002, Rawat2014}.

\begin{itemize}
	\item \textit{Sistema de alimentação.} Responsável pelo fornecimento de energia aos demais componentes, deve realizar a regulação de voltagem e outras adequações que se façam necessárias nas linhas de alimentação do nodo. Pode ainda realizar a captura e o armazenamento da energia proveniente de fontes secundárias, quando disponíveis. É essencialmente composto por uma ou mais baterias primárias e um conversor DC-DC, capaz de manter, de forma eficiente, os níveis de voltagem adequados nas linhas de alimentação.
	\item \textit{Sistema de aquisição de dados.} É capaz de obter dados amostrais a partir dos fenômenos monitorados, quando solicitado. Deve contar com um ou mais transdutores, responsáveis por converter as grandezas físicas monitoradas em sinais elétricos analógicos. E com circuitos de conversão analógico-digital (ADC's), cujo propósito é realizar a conversão dos sinais analógicos, provenientes dos transdutores, em sinais digitais compatíveis com as interfaces do sistema de processamento.
	\item \textit{Sistema de processamento.} É responsável por coordenar as tarefas delegadas ao nodo como, por exemplo, a obtenção de dados amostrais ou o roteamento de pacotes de dados encaminhados por outros nodos. Este sistema é composto essencialmente de um microcontrolador com memória integrada e, opcionalmente, pode contar com módulos de memória adicionais, a fim de permitir o armazenamento de maiores quantidades de dados localmente. O que pode vir a ser necessário de acordo com a topologia da rede ou caso não haja disponibilidade de conexões com o exterior.
	\item \textit{Sistema de comunicação.} É o responsável por permitir que os dados trafeguem entre os nodos que compõem a rede. Sua implementação mais comum, e a que será considerada ao longo deste trabalho, se baseia na utilização de radiotransmissores de baixa potência. No entanto vale ressaltar que as possibilidades de implementação não se restringem somente a este método. Em casos específicos, e onde a transmissão via rádio não é vantajosa, factível ou financeiramente viável, outros métodos de comunicação, como infra-vermelho ou transmissores acústicos, podem ser utilizados.
\end{itemize}

Algumas aplicações podem ainda requerer sistemas adicionais, como, por exemplo: sistemas de geolocalização (capazes de determinar a localização geográfica de determinado nodo); sistemas de mobilização (capazes de mover os nodos equipados entre diferentes posições, ou ainda, de reorientar antenas direcionais); sistemas de localização relativa (capazes de determinar a posição de um nodo em relação aos demais). A implementação destes sistemas pode tanto requerer hardware adicional ou permitir que sejam implementados puramente em software. No entanto, por se tratarem de componentes raramente utilizados, não serão considerados parte integrante do modelo básico, definido anteriormente.

Existem ainda os \textit{gateways} e \textit{sinks}, cuja arquitetura básica se assemelha à presente nos demais nodos. No entanto \textit{gateways} devem contar com interfaces de comunicação secundárias, capazes de estabelecer conexão entre a rede de sensores e alguma rede externa, ou diretamente com a internet. Já os \textit{sinks}, devem ser capazes de armazenar maior quantidade de dados localmente. Ambos frequentemente possuem maior poder de processamento, a fim de permitir que lidem com a carga extra de trabalho sem criar gargalos. Geralmente as funções de \textit{sink} e \textit{gateway} são atribuídas aos mesmos nodos, facilitando a coleta ou transferência dos dados coletados.

Para melhor compreender os desafios com os quais as propostas realizadas nas seções \ref{phynode} e \ref{phymac} visam lidar, certos aspectos específicos da operação dos nodos devem ser esclarecidos. A seguir serão melhor discutidas algumas peculiaridades e métodos relacionados à conservação de energia, protocolos de roteamento e estimativa de qualidade dos links de rádio, nas redes de sensores sem fios. Serão também analisadas iniciativas realizadas no sentido de fornecer padrões de implementação e operação.

\section{Conservação de energia}
A distribuição do consumo energético dentre os diferentes subsistemas, bem como as rotinas empregadas para transmissão e aquisição de dados, estão parcialmente sujeitas aos requisitos específicos das aplicações. No entanto algumas observações gerais são aplicáveis à maioria dos sistemas \cite{Anastasi2009}.

\begin{itemize}
	\item O consumo apresentado pelos sistemas de comunicação é muito superior ao originado no sistema de processamento. Sendo a energia utilizada para transmitir um único bit via rádio equivalente à empregada no processamento de até mil instruções, num nodo típico \cite{Pottie2000}. Sendo assim, é mais vantajoso aplicar processamento local, a fim de reduzir o volume de dados a ser transmitido, do que transmitir grandes quantidades de dados brutos.
	\item O consumo energético apresentado pelos transmissores de rádio nos modos de recepção, transmissão e espera é da mesma ordem. Existem, no entanto, em grande parte dos transmissores atuais, modos de baixo consumo, cuja utilização geralmente resulta em certa latência adicional, porém devem ser utilizados sempre que os protocolos de rede permitirem.
	\item O consumo do sistema de aquisição de dados é altamente dependente dos requisitos da aplicação, sendo em muitos casos significativo, e até mesmo superior ao consumo do sistema de comunicação. Nestes casos, torna-se interessante utilizar algoritmos destinados especificamente a minimizar o consumo referente à coleta de dados.
\end{itemize}

\begin{figure}[htb]
	\caption{\label{fig_energy_conservation}Técnicas para conservação de energia}
	\begin{center}
		\includegraphics[width=\linewidth]{EnergyConservation}
	\end{center}
	\legend{Adaptado de \citeonline{Anastasi2009}}
\end{figure}

Baseando-se nestas premissas é possível explorar diversas abordagens para otimizar o consumo de energia nos nodos de sensor. A \autoref{fig_energy_conservation} apresenta uma classificação hierárquica destas técnicas, como proposta em \cite{Anastasi2009}. Embora mais de uma dezena de técnicas específicas sejam apresentadas, conceitualmente estas abordagens podem ser todas classificadas em três subcategorias principais: técnicas baseadas em \textit{duty-cycling}, técnicas orientadas a dados e técnicas orientadas a mobilidade.

\subsection{Técnicas baseadas em \textit{duty-cycling}}

Técnicas baseadas em \textit{duty-cycling} consistem em estabelecer estratégias para que determinados sistemas dos nodos possam, periodicamente, ser desativados ou colocados em estados de baixo consumo. Possibilitam desta maneira que haja uma redução na média do consumo energético proveniente do sistema em questão. O período de atividade, ou \textit{duty-cycle}, é um valor percentual que representa a fração de tempo durante a qual este subsistema deverá manter-se ativo dentro de um determinado período de tempo.

Embora também possam ser aplicadas aos demais sistemas, técnicas desta natureza são comumente utilizadas para o controle do sistema de comunicação, dado que neste caso podem proporcionar melhorias significativas devido ao alto consumo deste sistema. Contudo o aumento na latência dos links de comunicação, ocasionado pela indisponibilidade dos transceptores durante os períodos em que são colocados em modos de baixo consumo, faz com que esta abordagem não seja adequada em todas as situações. Convenientemente em alguns casos é possível variar o \textit{duty-cycle} dinamicamente, de acordo com a intensidade do tráfego, por exemplo, a fim de proporcionar um melhor comprometimento entre a redução do consumo de energia e a latência dos links de comunicação.

\subsection{Técnicas orientadas a dados}

Apesar de as técnicas baseadas em \textit{duty-cycling} permitirem certa conservação de energia ao impedir que certos sistemas do nodo permaneçam em modos de alto consumo durante períodos de inatividade, por definição elas não dispõem dos meios para análise dos dados coletados pelo sensores ou retransmitidos. Assim sendo, ignoram quaisquer redundâncias presentes nas informações que circulam pela rede, potencialmente permitindo que alguma energia seja desperdiçada durante a manipulação de dados essencialmente desnecessários. Já as técnicas orientadas a dados são especificamente destinadas a analisar estes dados, de forma a eliminar possíveis redundâncias e evitar ciclos de amostragem ou retransmissões desnecessárias. Estas técnicas operam na camada de aplicação, sendo geralmente utilizadas em conjunto com outros métodos, operando em camadas inferiores, a fim de reduzir ainda mais o consumo energético das redes de sensores.

Mais especificamente, a ausência de métodos destinados ao controle de redundâncias afeta a eficiência energética da rede permitindo que dois tipos distintos de evento ocorram:

\begin{itemize}
	\item \textit{Transmissão de dados redundantes.} Dados amostrais geralmente apresentam forte correlação espacial e temporal logo, para satisfazer os requisitos de determinadas aplicações, pode não ser necessário transmitir na íntegra todas as amostras coletadas. É possível ainda aplicar técnicas de agregação e compressão de dados a fim de reduzir o volume de tráfego sem omitir amostras já coletadas.
	\item \textit{Aquisição de dados redundantes.} Alguns sistemas de aquisição apresentam alto consumo de energia, nestes casos o ideal é evitar, sempre que possível, que sejam realizados ciclos de amostragem desnecessários. Para tal geralmente são empregados algoritmos especializados, responsáveis por verificar se dados equivalentes foram previamente coletados (localmente ou por nodos próximos) antes que se iniciem novas amostragens.
\end{itemize}

\subsection{Técnicas orientadas à mobilidade}

Existem ainda técnicas de conservação de energia baseadas na mobilidade de nodos. Durante sua operação de rotina, os nodos de uma rede de sensores sem fios conduzem os dados coletados em direção aos \textit{sinks}. A principal consequência deste paradigma é a sobrecarga dos meios de transmissão dos nodos localizados próximo aos \textit{sinks}, através dos quais deverão ser roteados na íntegra os dados coletados pela rede \cite{Li2007}. A sobrecarga dos meios de transmissão nesta região provoca um consumo acelerado das reservas de energia dos nodos envolvidos: em geral os protocolos de comunicação se tornam menos eficientes em regiões sobrecarregadas, devido à possibilidade de ocorrerem colisões e, consequentemente, tentativas de retransmissão. A exaustão prematura das fontes de energia destes nodos, por sua vez, pode levar à subutilização de nodos mais distantes que, apesar de não terem suas reservas energéticas esgotadas, são inutilizados caso deixem de dispor de rotas de acesso aos \textit{sinks}.

Neste contexto, nodos ou \textit{sinks} móveis podem ser utilizados a fim de mitigar a sobrecarga de certas regiões, realizando a coleta dos dados enquanto se movimentam através da rede. Esta abordagem exige que os nodos estáticos fiquem à espera da passagem dos nodos móveis, quando terão a oportunidade de transmitir os dados diretamente a um \textit{sink} móvel, ou através de rotas mais eficientes. Este processo beneficia po balanceamento de cargas e possibilita a utilização uniforme das reservas de energia de todos os nodos estáticos. Ao eliminar regiões potencialmente sobrecarregadas é possível ainda obter algum aumento na eficiência dos protocolos de comunicação, devido à redução dos níveis de contenção (tentativas frustradas de acesso aos meios de transmissão). Para que estas abordagens sejam efetivas, no entanto, deve-se planejar cuidadosamente os padrões de movimentação dos nodos móveis. Caso se torne muito complexo ou dispendioso projetar nodos móveis customizados, existe ainda a possibilidade de utilizar-se de entidades que já se movimentem pela área monitorada, como veículos ou animais.

\section{Estimativa de qualidade  dos links}

A propagação dos sinais de rádio, principalmente nos links de baixa potência como os utilizados nas redes de sensores sem fios, é significativamente afetada por diversos fatores externos que contribuem para sua degradação. O que faz com que estes links se comportem de forma errática, sendo que sua performance pode variar ao longo do tempo, e de acordo com a localização dos nodos. Falhas de transmissão, como o corrompimento ou perda dos pacotes de dados e as ações subsequentes, destinadas à correção destas falhas, chegam a ser responsáveis por 50\% a 80\% do consumo de energia proveniente dos sistemas de comunicação, variando de acordo com a intensidade do tráfego \cite{Srinivasan2006}. Além de prejudicar a execução de protocolos de mais alto nível, uma vez que a largura de banda disponível é efetivamente reduzida e inviabiliza-se transações que imponham restrições temporais rigorosas.

As perturbações responsáveis por degradar a qualidade dos links são causadas por fenômenos cujo comportamento e origens são bastante distintos. No entanto podem ser classificadas quanto à sua natureza em duas principais categorias: atenuação e interferência. Entende-se por atenuação a absorção ou reflexão parcial, em decorrência da interação com o ambiente, dos sinais de rádio emitidos por um transmissor. O que resulta numa redução da intensidade do sinal captado pelo receptor. Já o termo interferência se refere ao acoplamento de sinais externos, e de frequência similar, às transmissões realizada entre um transmissor e um receptor. Este processo causa distorções no sinal original e resulta no corrompimento dos dados transmitidos sem que haja, necessariamente, qualquer redução na intensidade do sinal captado pelo receptor. De acordo com \cite{Baccour2012} são três os principais fatores que exercem influência sobre a qualidade dos links de rádio:
\begin{itemize}
	\item \textit{O ambiente.} Além de ser o principal componente para a atenuação dos sinais, também é responsável pela ocorrência do fenômeno conhecido como propagação multicaminho. Que ocorre quando, ao se propagar através de certo ambiente, componentes parciais de um mesmo sinal são refletidos e dispersam-se através de múltiplas rotas, cada um deles atingindo o receptor com diferentes intensidades e ligeira defasagem temporal \cite{Kusy2011}. O que resulta na distorção do sinal original e leva ao corrompimento dos dados. O ambiente também determina os níveis de ruído de fundo, contribuindo para determinação da relação entre sinal e ruído.
	\item \textit{O hardware.} O hardware dos transmissores invariavelmente apresenta alguma variabilidade, devido a seu próprio ruído interno e às faixas de tolerância aplicadas aos valores dos componentes utilizados para sua construção \cite{Goldsmith2005}. Com o desenvolvimento de hardware mais atualizado, este fator tende a exercer menor influência na qualidade final dos links, porém não deve ser desconsiderado durante a concepção de métodos de estimativa de qualidade ou protocolos de comunicação. 
	\item \textit{Interferências.} São causadas pela presença de múltiplos sinais sendo transmitidos concorrentemente, num mesmo ambiente, através de faixas de frequência similares. Podendo inclusive estes sinais originarem-se em diferentes nodos de uma mesma rede de sensores sem fios. Segundo \cite{Baccour2012} a área sobre a qual transmissões de um nodo podem causar interferência destrutiva supera a área na qual este nodo consegue manter links de comunicação confiáveis, logo podem-se estabelecer relações de interferência mesmo entre nodos que não sejam vizinhos diretos. Fontes de interferência podem ainda incluir outros equipamentos que compartilhem as mesmas faixas de frequência para comunicação sem fios e ruído eletromagnético gerado involuntariamente por certos equipamentos elétricos.
\end{itemize}

Nas redes de sensores sem fios, ao contrário do que acontece em outros ambientes, as aplicações finais não podem desacoplar completamente suas atividades dos processos de comunicação. Estas aplicações e os protocolos de rede por elas empregados devem, a fim de otimizar seus processos, levar em consideração a qualidade dos links disponíveis. Estimativas de qualidade acuradas podem auxiliar no processo de seleção de rotas mais eficientes, caso existam \cite{YanjunLi2005}. Ou ainda serem utilizadas para determinar os momentos mais apropriados para transmissão de dados, nos quais haverá maior probabilidade de sucesso \cite{Srinivasan2008}. Alternativamente, quando disponibilizadas a processos de mais alto nível, estas estimativas podem também ser utilizadas para adequar certas rotinas à qualidade momentânea dos links. Permitindo por exemplo alterações dinâmicas nas frequências de aquisição de dados, ou no comportamento de mecanismos de compressão ou agregação de dados, que podem se adequar à banda disponível.

\subsection{Etapas do processo de estimativa}

Entende-se por estimativa da qualidade de um link como a avaliação de uma métrica, definida matematicamente a partir de parâmetros relevantes, e aplicada sobre um conjunto de dados amostrais. O qual pode ser composto por uma quantidade predefinida de amostras, ou por uma quantidade variável de amostras coletadas durante um período predefinido de tempo. Geralmente refere-se a esta métrica como LQE (\textit{Link Quality Estimator}). Para que seja possível realizar a estimativa é necessário que sejam realizadas medições no link, por exemplo, através do registro dos números de sequência dos pacotes recebidos ou da aferição da intensidade do sinal durante a recepção dos pacotes de dados. É necessário ainda que se defina um método de monitoramento, responsável por garantir que haverá uma quantidade mínima de tráfego no link, a fim de permitir que as medições possam ser realizadas com frequência regular. O processo de estimativa de qualidade consiste, portanto, em três atividades distintas: monitoramento do link, medições no link e avaliação da métrica \cite{Baccour2012}. Estas atividades são ilustradas na \autoref{fig_lqe_steps}.

\begin{figure}[htb]
	\caption{\label{fig_lqe_steps}Processo de estimativa de qualidade do link.}
	\begin{center}
		\includegraphics[width=\linewidth]{WSN_LQE_Steps}
	\end{center}
	\legend{Adaptado de \citeonline{Baccour2012}}
\end{figure}

A primeira atividade, monitoramento do link, pode ser realizada baseando-se em três diferentes abordagens: monitoramento ativo, passivo ou híbrido. As informações obtidas pelos mecanismos de monitoramento, além de permitirem a coleta de dados para estimativa de qualidade dos links, podem ser consumidas por outros algoritmos relacionados aos processos de roteamento ou controle de topologia.

O monitoramento ativo é realizado através da troca de pacotes de teste, denominados \textit{beacons}, entre os nodos e seus vizinhos. Estes pacotes podem ser transmitidos via \textit{broadcast} \cite{DeCouto2005} ou \textit{unicast} \cite{Kim2006}. Mecanismos baseados em \textit{broadcast} tem a vantagem de não requererem \textit{feedback}, por parte dos destinatários, quanto ao recebimento dos \textit{beacons}. O que permite implementações mais simples, se comparado aos mecanismos baseados em \textit{unicast}, e evita possíveis retransmissões automáticas por parte da camada de \textit{link}. Já as alternativas baseadas em \textit{unicast} proporcionam a obtenção de dados mais acurados, uma vez que utilizam pacotes semelhantes ao tráfego real \cite{Sang2010} e são capazes de identificar possíveis assimetrias nos links. Entretanto mecanismos baseados em \textit{unicast} tendem a consumir uma quantidade maior de recursos do sistema de comunicação.

O monitoramento passivo explora o tráfego usual da rede para computar as estatísticas de conectividade. Com isto reduz-se a utilização de banda e o consumo de energia associados à transmissão dos \textit{beacons}. Caso o tráfego envolvendo um único nodo não seja suficiente para manter suas estimativas atualizadas pode-se utilizar a técnica conhecida como \textit{sniffing}, que consiste em capturar todos os pacotes que trafegam nas proximidades, permitindo a obtenção de um volume maior de informações \cite{LaI2003,Woo2003}. É possível ainda capturar somente as confirmações de recebimento emitidas por nodos vizinhos, aproveitando-se de \textit{beacons} emitidos por outros nodos \cite{PengJiang2006,YanjunLi2005}. A utilização de técnicas de \textit{sniffing}, no entanto, exige que os módulos de rádio sejam mantidos em modo de escuta por períodos maiores, elevando o consumo energético do sistema de comunicação \cite{LaI2003}. Adicionalmente, em redes onde o volume de tráfego é reduzido ou desbalanceado, o monitoramento passivo pode criar uma deficiência de medições atualizadas, levando a estimativas de qualidade imprecisas \cite{Baccour2012}.

Existe ainda a possibilidade de utilizar métodos que combinem as técnicas de monitoramento ativo e passivo, permitindo melhores relações entre consumo de energia e obtenção de estimativas atualizadas \cite{Gnawali2009,Kim2006}. Geralmente estas técnicas, denominadas híbridas, consistem em utilizar o monitoramento passivo durante os períodos em que o volume de tráfego real permitir a obtenção de medições suficientemente atualizadas, alternando para mecanismos de monitoramento ativo quando o tráfego preexistente não for suficiente para manter as estimativas precisas para todos os nodos.

A segunda atividade relacionada ao processo de estimativa de qualidade consiste em realizar medições a partir dos dados obtidos na etapa anterior. As atividades realizadas nesta etapa envolvem computar parâmetros que não podem ser obtidos diretamente durante o monitoramento dos links, porém podem ser deduzidos a partir dos dados previamente coletados. Parâmetros como, por exemplo, a relação sinal-ruído (\textit{Signal to Noise Ratio} ou SNR) e taxas de recebimento ou perda de pacotes, podem ser computados nesta faze. É importante notar que os parâmetros monitorados pelos emissores e receptores, através das atividades de monitoramento, não são os mesmos. Contadores de retransmissão, por exemplo, estão disponíveis somente para emissores. Já medições de intensidade do sinal recebido estão presentes somente nos receptores. Portanto técnicas de medição diferentes devem ser aplicadas a cada uma das extremidades do link.

A última etapa deste processo consiste em utilizar os dados provenientes da atividade de medição para avaliação de uma métrica (LQE), através da qual obtém-se a estimativa de qualidade do link. Os LQEs mais básicos podem ser subdivididos em duas categorias: baseados em software ou hardware. LQEs baseados puramente em hardware geram estimativas a partir de parâmetros associados à integridade do sinal, os mais comuns são: SNR, RSSI (\textit{Received Signal Strength Indicator}) e LQI (\textit{Link Quality Indicator}). Já os LQEs baseados puramente em software como, por exemplo, PRR (\textit{Packet Reception Ratio}) ou RNP (\textit{Required Number of Packets}), avaliam parâmetros relacionados à integridade dos pacotes recebidos \cite{Baccour2012}.

No entanto a confiabilidade dos estimadores citados acima é bastante limitada, uma vez que consideram somente uma pequena quantidade de parâmetros como base para o cálculo dos resultados. A fim de obter estimativas mais acuradas em geral são utilizados LQEs mais abrangentes, capazes de considerar simultaneamente parâmetros provenientes tanto do hardware quanto do software. \textit{Four-bit}, um LQE proposto em \citeonline{Fonseca2007}, por exemplo, considera parâmetros provenientes das camadas de rede, link e física. Já \textit{Triangle Metric}, propsoto em \citeonline{Boano2010}, considera parâmetros provenientes tanto da camada física quanto da camada de controle de acesso ao meio. Há ainda LQEs como \textit{MetricMap}, proposto em \citeonline{Wang2007}, que utilizam técnicas de aprendizado de máquina, lógica \textit{fuzzy} e outras técnicas avançadas de processamento para aprimorar seus resultados.

\subsection{Requisitos para estimativa de qualidade dos links}

Certos requisitos devem ser satisfeitos para que um estimador de qualidade possa operar de forma eficiente:

\begin{enumerate}
	\item \textit{Eficiência energética.} Uma vez que considera-se o consumo de energia como sendo uma grande preocupação nas redes de sensores sem fios, estimadores de qualidade dos links não devem sobrecarregar os sistemas de processamento e devem requerer o mínimo possível de comunicação adicional. Sistemas ativos com altas taxas de amostragem, portanto, devem ser evitados.
	\item \textit{Acurácia.} Entende-se por acurácia como sendo a habilidade de um LQE para caracterizar corretamente o comportamento do link. Sua acurácia determinará a eficiência com que um LQE será capaz de otimizar a utilização dos links pelos protocolos de rede. Na teoria da estimativa define-se que um processo estimado possa ser comparado a um processo conhecido e real, através de técnicas estatísticas, devendo apresentar comportamentos similares. No entanto, tal comparação não é possível para a estimativa de qualidade dos links pois: (i) não há nenhuma métrica amplamente adotada como medida absoluta da qualidade de um link e (ii) a qualidade do link pode ser representada por quantias de diferentes naturezas: alguns estimadores baseiam-se na cálculo da taxa de recepção de pacotes, outros na contagem do número de retransmissões necessárias, entre outros. 

	Contudo a acurácia de um LQE ainda pode ser avaliada indiretamente, isto é, utilizando-se métricas subordinadas aos efeitos da estimativa sobre a eficiência dos protocolos de rede \cite{Baccour2012}. Por exemplo, em \citeonline{Fonseca2007} são avaliados os efeitos da utilização do estimador \textit{Four-bit} sobre a performance do protocolo de coleta de dados CTP (\textit{Collection Tree Protocol}). Conclui-se que a utilização deste estimador promove melhorias na taxa de entrega de pacotes, comprovando que o estimador favorece a seleção consistente de links de melhor qualidade, logo possui boa acurácia. Uma outra abordagem é utilizada em \citeonline{Baccour2011}, onde a distribuição temporal das estimativas fornecidas por alguns estimadores é estatisticamente analisada a fim de determinar sua confiabilidade.
	\item \textit{Reatividade.} Considera-se como reatividade, a capacidade de um estimador para detectar e reagir rapidamente a mudanças perenes na rede \cite{KimM.2001}. Um LQE reativo permite que os protocolos de roteamento e o controle de topologia se adaptem rapidamente a alterações na conectividade dos nodos. O nível de reatividade de um LQE depende de dois fatores: (i) o conjunto de dados utilizados para estimativas e (ii) o método de monitoramento escolhido. Conjuntos de dados pequenos e métodos de monitoramento ativos, utilizando altas taxas de amostragem levam a LQEs reativos. No entanto é importante notar que alguns LQEs são naturalmente mais reativos que outros. LQEs avaliados no emissor demonstraram-se geralmente mais reativos que LQEs avaliados no receptor \cite{Baccour2011}.
	\item \textit{Estabilidade.} Refere-se a estabilidade como sendo a capacidade de um LQE tolerar transientes na qualidade dos links. É importante que estes transientes sejam filtrados a fim de evitar que protocolos de roteamento e controle de topologia executem, desnecessariamente, rotinas custosas destinadas a reorganizar a rede após alterações perenes na rede como, por exemplo, a reinicialização das  tabelas de roteamento. Estimadores estáveis podem ser obtidos através do monitoramento do link a longo prazo \cite{Lin2009}. A estabilidade pode ser aferida através do coeficiente de variação das estimativas de qualidade, computado através da razão entre o desvio padrão das estimativas e sua média \cite{Woo2003}. Ou ainda avaliando-se o impacto do LQE sobre as rotinas de roteamento. Tipicamente LQEs estáveis resultam em topologias estáveis, nas quais observam-se poucas reorganizações de rotas \cite{Baccour2009}.
\end{enumerate}

Para que um estimador opere de forma eficiente, deve adotar a relação apropriada entre estabilidade e reatividade, permitindo que o roteamento adote a utilização dos links de comunicação que apresentem melhor qualidade a curto e longo prazo. Geralmente estimadores de melhor desempenho combinam múltiplas métricas para obter esta relação \cite{Sang2010}, ou utilizam para tal uma modelagem matemáticas precisa do comportamento da rede \cite{Woo2003}.

\subsection{Mecanismos básicos para LQE}

A seguir serão apresentados alguns mecanismos básicos, utilizados para avaliação de diferentes características dos \textit{links} de rádio, através dos quais compõem-se sistemas de estimativa mais complexos. Porém, antes de entrar em detalhes a respeito destes mecanismos, é importante compreender quais são os fenômenos que eles buscam avaliar e como interferem na qualidade das comunicações entre nodos. Em \cite{Kusy2011} encontram-se destacados alguns modos de falha mais comuns nos links de baixa potência utilizados em RSSFs.

\begin{itemize}
	\item Falha por parte do destinatário, ao detectar o preâmbulo de uma transmissão iniciada por outro nodo. Geralmente devido a uma instabilidade temporária no link, o destinatário pode não captar o preâmbulo, elemento responsável por sinalizar o início da transmissão de um pacote. Neste caso o pacote é completamente ignorado por seu destinatário e será, possivelmente, retransmitido.
	\item Falha, por parte do destinatário, ao capturar corretamente os dados de um pacote durante seu recebimento. Geralmente ocorre devido à instabilidade no link durante a transmissão. Neste caso o pacote será recebido, porém irá falhar na verificação de integridade, e será posteriormente descartado. A confirmação de recebimento não será enviada, forçando o emissor a retransmitir o pacote.
	\item Falha, por parte do emissor, ao captar a confirmação de recebimento de um pacote previamente enviado. Neste caso o destinatário recebe e processa o pacote corretamente, no entanto alguma instabilidade afeta o link no momento em que a confirmação de recebimento está sendo transmitida. Na ausência de uma confirmação, o emissor conclui erroneamente que o pacote não foi recebido corretamente. Geralmente são realizadas novas tentativas de transmissão.
\end{itemize}

\begin{figure}[htb]
	\caption{\label{fig_lqe_methods}Métodos para estimativa de qualidade dos links.}
	\begin{center}
		\includegraphics[width=\linewidth]{WSN_LQE_Methods}
	\end{center}
	\legend{Adaptado de \citeonline{Baccour2012}}
\end{figure}

Embora somente detectar e quantificar a presença de componentes de interferência, atenuação ou distorções causadas pela propagação multi-caminho sobre os sinais subjacentes não seja suficiente para estimar a qualidade de um link com a competência necessária para evitar os modos de falha citados anteriormente, estas informações são a base para avaliação de métricas mais completas. Os mecanismos utilizados para este fim podem ser divididos em duas categorias básicas: baseados em \textit{hardware} e baseados em \textit{software}. Encontra-se ilustrada na \autoref{fig_lqe_methods} uma classificação para estes mecanismos. Em geral os mecanismos baseados em hardware são capazes de prover informações acuradas sobre determinadas propriedades físicas dos sinais recebidos, como intensidade do sinal (RSSI) ou relação sinal-ruído (SNR). No entanto não são adequados para avaliar o impacto destes componentes sobre a eficiência do link. Já os mecanismos baseados em software podem prover índices de mais alto nível, como contadores de retransmissão (RNP) ou taxa de recepção de pacotes (PRR). Contudo não são capazes de apontar os fatores de baixo nível responsáveis pelos resultados obtidos.

\subsubsection{Mecanismos baseados em hardware}

Os mecanismos de estimativa baseados em hardware provêm informações a respeito da qualidade dos sinais enquanto processados pelas camdas inforiores da pilha de protocolos. A implementação de tais mecanismos depende de hardware dedicado, geralmente integrado ao \textit{frontend} analógico dos \textit{chips} de rádio. São três os principais mecanismos desta categoria: RSSI, SNR e LQI.

\begin{itemize}
	\item \textit{Radio Signal Strength Indicator (RSSI).} O RSSI provê uma medida instantânea da intensidade com que um sinal é recebido pelo módulo de rádio. Quando não há nenhuma transmissão em curso representa uma estimativa da intensidade do ruído de fundo \cite{Baccour2012}. A utilização do RSSI é atrativa por consistir numa forma simples e acessível de avaliar a qualidade do link entre dois nodos. No entanto, apesar de existir alta correlação entre baixa intensidade de sinal e falhas de transmissão, a afirmativa complementar pode não ser verdadeira. Mesmo links que possuam RSSI consistentemente elevados podem apresentar altas taxas de falha e perdas de pacotes \cite{Srinivasan2006}. Adicionalmente os valores obtidos a partir deste mecanismo são de natureza arbitrária, podendo variar de acordo com a implementação utilizada por cada fabricante.
	\item \textit{Signal to Noise Ratio (SNR).} Esta métrica representa a intensidade do sinal recebido em relação ao nível de ruído de fundo. Isto a torna uma métrica mais acurada que o RSSI que, por sua vez, representa a intensidade do sinal recebido somada ao ruído de fundo. Sendo assim, SNR é mais capaz de estimar quais as chances de o sinal recebido ser decodificado corretamente, não importando sua intensidade \cite{Srinivasan2010}. No entanto o SNR só pode ser calculado a partir de pacotes detectados com sucesso, o que faz com que não seja confiável quando aplicado aos nodos de regiões transicionais \cite{LaI2003}. Nestas regiões os valores obtidos para o SNR podem ser tendenciosos, superestimando a qualidade do link, uma vez que não contabilizam os pacotes perdidos.
	\item \textit{Link Quality Indicator (LQI).} É uma métrica proposta no padrão IEEE 802.15.4 \cite{IEEE2003}, porém sua implementação pode variar de acordo com o fabricante e o \textit{chipset} escolhidos. O LQI pode indicar se um link apresenta excelente qualidade ou não. No entanto não é um bom indicador para links intermediários, apresentando alta variância nesta situações \cite{Baccour2012}. Aconselha-se utilizar a média dos resultados obtidos durante algum período para melhorar a qualidade das estimativas obtidas \cite{Srinivasan2006}. É possível ainda basear-se no LQI para estimar a qualidade do link indiretamente, utilizando sua variância como métrica final \cite{Boano2009}. Links de baixa qualidade apresentarão alta variância e links de boa qualidade apresentarão baixa variância. Esta técnica permite uma resposta mais rápida que o cálculo da média, pois permite que um resultado final seja obtido sem a necessidade de avaliar uma quantidade tão grande de pacotes.
\end{itemize}
	
	Todos estes mecanismos, no entanto, compartilham algumas limitações. Primeiramente, só podem ser computados a partir de pacotes recebidos com sucesso; logo, tendem a superestimar a qualidade dos links que apresentem muitas perdas de pacotes, uma vez que as estatísticas referentes a estes pacotes são ignoradas. E adicionalmente, apesar de consistirem em métricas rápidas e de baixo custo para classificar links como bons ou ruins, não são capazes de fornecer estimativas mais refinadas \cite{Baccour2012}, geralmente não sendo suficientes para prover informações relevantes a respeito dos links situados nas regiões transitórias.
	
\subsubsection{Mecanismos baseados em software}

Os mecanismos baseados em software provêm informações a respeito do link a partir das camadas superiores da pilha de protocolos. São mais adequados que os mecanismos baseados em hardware para computar estatísticas globais, porém menos capazes de avaliar as condições instantâneas dos meios de transmissão. São três as principais abordagens para desenvolvimento dos mecanismos pertencentes a esta categoria: PRR, RNP, pontuação.

\begin{itemize}
	\item \textit{Packet Reception Ratio (PRR).} Representa a razão entre o número de pacotes recebidos com sucesso e o número total de pacotes transmitidos. Em alguns casos também é utilizada a métrica análoga e complementar PER (\textit{Packet Error Rate}), obtida a partir de: $PER = 1 - PRR$ \cite{Baccour2012}.
	\item \textit{Required Number of Packets (RNP).} Esta métrica é obtida através da contagem do número necessário de retransmissões realizadas para que cada pacote enviado seja entregue com sucesso ao destinatário. RNP assume retransmissões automáticas, por parte dos protocolos de roteamento, caso pacotes sejam perdidos \cite{Cerpa2005}. Uma abordagem semelhante é utilizada na métrica \textit{Acknowledge Reception Radio (ARR)}, que considera a relação entre o número total de pacotes transmitidos e o número de confirmações de recebimento coletadas \cite{Baccour2012}.
	\item \textit{Métricas baseadas em pontuação.} Existem ainda métricas que apresentam seus resultados através de faixas de valor definidos arbitrariamente, sem que estejam diretamente relacionados com algum fenômeno físico \cite{Baccour2012}.
\end{itemize}

\section{Iniciativas de padronização}
A norma IEEE 802.15.4 \cite{IEEE2003}, define um protocolo de comunicação e parâmetros de implementação da camada física para interconexão de dispositivos via radiofrequência em redes de área pessoal sem fio com baixas taxas de transmissão (LR-WPANs). Esta norma é base para alguns dos protocolos mais disseminados na indústria como ZigBee \cite{bibid}, 6LowPAN \cite{bibid}, WirelessHART \cite{bibid}, MiWi \cite{bibid} entre outros. A norma original, publicada em 2003, propunha apenas duas implementações distintas para a camada física (PHY), sendo a primeira destinada à operação nas bandas de 868 MHz e 915 MHz utilizando modulação BPSK e a segunda na banda de 2450 MHz, através da modulação O-QPSK. Desde sua publicação original, no entanto, esta norma recebeu diversas emendas e revisões, através das quais foram adicionados novos PHYs e realizadas alterações nos protocolos de comunicação. 

Dispositivos compatíveis devem implementar por completo ao menos um PHY, podendo opcionalmente ser compatíveis com múltiplos PHYs. A seleção dos PHYs a serem utilizados, no entanto, fica a cargo das regulamentações locais e preferências do usuário. Em sua última versão, IEEE 802.15.4-2015 \cite{IEEE2016}, há dezenove PHYs, capazes de operar em bandas de frequência que variam entre 169 MHz e cerca de 9 GHz. A grande diversidade de  implementações prevista pela norma faz com que frequentemente as legislações vigentes em cada país sejam compatíveis, simultaneamente, com diversas delas.
 % ---
Cada PHY define uma modulação específica, taxas de transferência distintas e um certo número de canais para cada banda de frequência suportada. Sendo que cada canal representa uma pequena faixa de frequência dentro do espectro da banda a qual pertence. A quantidade de canais disponíveis em cada PHY pode variar, de acordo com a largura de banda dedicada a cada um deles. 

A seleção do método de acesso ao meio físico na norma IEEE 802.15.4 é realizada através de um identificador único de 32 bits, composto por um número de página (armazenado nos 5 bits mais altos) e um número de canal (27 bits restantes). Esta arquitetura foi introduzida com o intuito de permitir a declaração de novos PHYs a medida que fossem realizadas novas revisões do protocolo. 

\section{Discussão}
Motes are equipped with a low-rate (10–100 kbps) and short-range (less than 100 m) wireless radio, e.g., IEEE 802.15.4 radio to communicate among themselves. 
Since radio communication consumes most of the power, the radio must incorporate energy-efficient communication techniques.
The power source commonly used is rechargeable batteries.
Since motes can be deployed in remote and hostile environments they must use little power and must employ built-in mechanisms to extend network lifetime.
For example, motes may be equipped with effective power harvesting methods, such as solar cells, so they may be left unattended for years \cite{Rawat2014}.

Devido às restrições na complexidade, no consumo energético e nos custos de produção, os transmissores de rádio utilizados para este fim geralmente operam com baixa potência e pequenas taxas de transmissão. Isso faz com que os links estabelecidos se tornem bastante susceptíveis à variações intermitentes de qualidade, especialmente nos casos em que compartilham sua banda de frequência com outros sistemas mais bem difundidos. Como é o caso da disputada faixa de 2.4GHz, na qual também estão outras tecnologias bem difundidas como WiFi e Bluetooth.


\chapter{Trabalhos relacionados}
\section{Diversidade de interfaces}

Nas seções anteriores foram discutidos os obstáculos impostos às redes de sensores sem fios pela instabilidade nos links de comunicação. Foram introduzidos mecanismos capazes de estimar a qualidade destes links e também protocolos de roteamento que, com auxílio destes mecanismos, são capazes de operar de foram eficaz neste ambiente. No entanto, os mecanismos e protocolos propostos têm uma deficiência em comum: podem promover melhorias na interação entre os demais sistemas e as interfaces de comunicação, mas não são capazes de melhorar efetivamente a qualidade das transmissões realizadas através destas interfaces. De fato, nenhum mecanismo baseado puramente em software será capaz de fazê-lo, uma vez que são as caraterísticas físicas dos meios de transmissão que têm maior impacto sobre seu comportamento.

Soluções físicas para a instabilidade dos links geralmente envolvem aumentar a densidade da rede ou a potência de transmissão dos nodos. No entanto, estes métodos resultam em provisionamento excessivo de recursos e nem sempre melhoram a confiabilidade das redes \cite{Kusy2011}.

Nesta seção será apresentada uma abordagem alternativa para este problema: a diversidade de interfaces. Esta abordagem consiste em utilizar múltiplas interfaces de comunicação independentes, e com características distintas. Entre seus principais benefícios está que a probabilidade de todas as interfaces falharem simultaneamente é menor do que a probabilidade isolada de uma delas falhar \cite{Jurdak2011, Kusy2011}. Há ainda o benefício de poder selecionar dinamicamente a interface mais adequada para cada transmissão \cite{Gummeson2010, Lymberopoulos2008, Pering2006}. Para interfaces de rádio, a diversidade pode ser obtida, por exemplo, utilizando-se bandas de frequência distintas ou antenas colocadas em posições diferentes \cite{Jurdak2011, Kusy2011}, tecnologias distintas, como 802.11, 802.15.4 ou Bluetooth \cite{Lymberopoulos2008, Pering2006}, ou ainda potências de transmissão distintas \cite{Gummeson2010}. A seguir serão analisados alguns trabalhos relacionados à diversidade de interfaces.

\section{Estimativa de qualidade dos links}
\section{Discussão}
% ---

% ---
% Materiais e Métodos
% ---
\chapter{PhyNode: plataforma de desenvolvimento} \label{phynode}

Dado que a utilização de diferentes modulações, ou bandas de frequência, resulta, de acordo com as condições do ambiente, em alterações na qualidade dos links de rádio \cite{bibid}, considera-se que seja possível explorar a diversificação de interfaces admitida pela norma a fim de otimizar a performance de links de comunicação. No caso de dispositivos dotados de múltiplas interfaces, uma forma através da qual este objetivo pode ser alcançado é utilizando um algoritmo de seleção dinâmica, capaz de eleger a interface mais adequada, a cada momento, para a transferência dos dados. Com este objetivo, nas seções seguintes, será proposta a arquitetura batizada de PhyNode, dotada de três interfaces IEEE 802.15.4, operando nas bandas de frequência de 433 MHz, 915 MHz e 2450 MHz. Selecionadas a fim de satisfazer as exigências da regulamentação local, de acordo com a resolução nº 452 da ANATEL \cite{bibid} e a fim de prover a maior amplitude espectral possível.

% ---

A plataforma PhyNode inclui três \textit{transceivers}, sendo dois CC1201 \cite{bibid} destinados às bandas de frequência sub-GHz e um CC2520 \cite{bibid}, destinado à banda de 2450 MHz. As configurações disponíveis em cada um destes componentes permitem que, em alguns casos, mais de um PHY seja utilizado através do mesmo \textit{transceiver}, porém  não simultaneamente.
	\section{Diversidade de Interfaces}
	PhyNode pode utilizar-se do identificador único de 32bits presente no padrão IEEE 802.15.4 para carregar as configurações adequadas num \textit{trasnceiver} antes de transmitir um pacote de dados. O conjunto de PHYs IEEE 802.15.4 suportados em PhyNode é apresentado na Tabela \ref{tab:modos_opr}.
	
	\begin{table}[h]
		\centering
		\begin{tabular}{|c|c|l|l|l|l|l|}
			\hline
			\textbf{Página}    & \multicolumn{1}{l|}{\textbf{Canais}}   & \textbf{Banda}          & \textbf{Modulação}   & \textbf{Taxa transf.} \\ \hline
			0                  & 11$\sim$26                             & 2450MHz                 & O-QPSK               & 250Kb/s               \\ \hline
			1                  & 1$\sim$10                              & 915MHz                  & ASK                  & 250Kb/s               \\ \hline
			\multirow{3}{*}{7} & \multirow{3}{*}{0$\sim$14}             & \multirow{3}{*}{433MHz} & \multirow{3}{*}{MSK} & 250Kb/s               \\
			&                                        &                         &                      & 100Kb/s               \\
			&                                        &                         &                      & 31.25Kb/s             \\ \hline
			9                  & 0$\sim$128                             & 915MHz                  & 2-FSK                & 50Kb/s                \\ \hline 
		\end{tabular}
		\caption{Modos de operação suportados}
		\label{tab:modos_opr}
	\end{table}
	
	As frequências dos canais disponíveis especificamente nos PHYs utilizados pelos módulos PhyNode são  descritas de acordo com as equações (\ref{eq:ch433msk}), (\ref{eq:ch915ask}), (\ref{eq:ch915fsk}) e (\ref{eq:ch2450oqpsk}).
	
	\begin{equation}
	\label{eq:ch433msk}
	f_{P0}(C) = 433.164 + 0.108\times C, \forall C \in \{0..14\}
	\end{equation}
	
	\begin{equation}
	\label{eq:ch915ask}
	f_{P1}(C) = 906 + 2 (C - 1), \forall C \in \{1..10\}
	\end{equation}
	
	\begin{equation}
	\label{eq:ch915fsk}
	f_{P7}(C) = 902.2 + 0.2 \times C, \forall C \in \{0..128\}
	\end{equation}
	
	\begin{equation}
	\label{eq:ch2450oqpsk}
	f_{P9}(C) = 2405 + 5 (C - 11), \forall C \in \{11..26\}
	\end{equation}
	
	Adicionalmente é possível selecionar a potência de transmissão utilizada por cada \textit{trasnceiver}, sendo que o modelo CC1201 é capaz de operar entre -40 DBm e 14 DBm (ou -37 DBm a 17 DBm, considerando-se o uso de antenas com ganho de 3 DBi), já o modelo CC2520 pode operar entre -18 DBm e 5 DBm (ou -13 DBm a 10DBm, considerando-se o uso de antenas com ganho de 5 DBi).
	
	Também é importante ressaltar que para que se possa estabelecer um link entre dois dispositivos eles devem estar alinhados quanto à diversos parâmetros, com exceção geralmente apenas da potência de transmissão. Para que estes parâmetros possam ser alterados dinamicamente é são requeridos mecanismos de sincronia, responsáveis manter ambas as configurações atualizadas em ambas as extremidades do link.
	\section{Parâmetros observáveis}
	CC2520
	\begin{itemize}
		\item EIRP = -13 ~ 10
		\item CHANNEL = 2405 ~2505 [16]
	\end{itemize}
	CC1201
	\begin{itemize}
		\item EIRP = -37 ~ 17
		\item CHANNEL = 410 ~ 475 [14]
		\item SYMBOL RATE
		\item MODULATION
	\end{itemize}
	Common
	\begin{itemize}
		\item Packet length
		\item Packet interval
	\end{itemize}
	Output parameters
	\begin{itemize}
		\item RSSI
		\item LQI
		\item CRC\_OK
	\end{itemize}
\section{Caracterização dos links}

\chapter{PhyMAC: Camada MAC multi-interface} \label{phymac}
% TODO PhyMAC

% ---
% Resultados
% ---
\chapter{Avaliação experimental}
% TODO Avaliação experimental
\chapter{Resultados}
% TODO Resultado
\chapter{Propostas para desenvolvimento futuro}
% TODO Desenvolvimentos futuros

% ----------------------------------------------------------
% Finaliza a parte no bookmark do PDF
% para que se inicie o bookmark na raiz
% e adiciona espaço de parte no Sumário
% ----------------------------------------------------------
\phantompart

% ---
% Conclusão
% ---
\chapter{Conclusão}
% TODO - Conclusão

% ----------------------------------------------------------
% ELEMENTOS PÓS-TEXTUAIS
% ----------------------------------------------------------
\postextual
% ----------------------------------------------------------

% ----------------------------------------------------------
% Referências bibliográficas
% ----------------------------------------------------------
\bibliography{dissert}

% ----------------------------------------------------------
% Glossário
% ----------------------------------------------------------
%
% Consulte o manual da classe abntex2 para orientações sobre o glossário.
%
%\glossary

% ----------------------------------------------------------
% Apêndices
% ----------------------------------------------------------

% ---
% Inicia os apêndices
% ---
%\begin{apendicesenv}

% Imprime uma página indicando o início dos apêndices
%\partapendices

% ----------------------------------------------------------
%\chapter{Quisque libero justo}
% ----------------------------------------------------------

%\lipsum[50]

%\end{apendicesenv}
% ---


% ----------------------------------------------------------
% Anexos
% ----------------------------------------------------------

% ---
% Inicia os anexos
% ---
%\begin{anexosenv}

% Imprime uma página indicando o início dos anexos
%\partanexos

% ---
%\chapter{Morbi ultrices rutrum lorem.}
% ---
%\lipsum[30]

%\end{anexosenv}

%---------------------------------------------------------------------
% INDICE REMISSIVO
%---------------------------------------------------------------------
\phantompart
\printindex
%---------------------------------------------------------------------

\end{document}
