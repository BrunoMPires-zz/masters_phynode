\documentclass{article}
\usepackage[utf8]{inputenc}
\usepackage{multirow}
\begin{document}
\title{Teste}
\author{Bruno Monteiro Pires}
\maketitle
\clearpage

\section{Introdução}



\section{Radio Parameters}
CC2520
\begin{itemize}
\item EIRP = -13 ~ 10
\item CHANNEL = 2405 ~2505 [16]
\end{itemize}
CC1201
\begin{itemize}
\item EIRP = -37 ~ 17
\item CHANNEL = 410 ~ 475 [14]
\item SYMBOL RATE
\item MODULATION
\end{itemize}
Common
\begin{itemize}
\item Packet length
\item Packet interval
\end{itemize}
Output parameters
\begin{itemize}
\item RSSI
\item LQI
\item CRC\_OK
\end{itemize}

\subsection{IEEE 802.15.4 compliant operating modes}
\begin{table}[h]
\centering
\begin{tabular}{|l|l|l|l|l|}
\hline
\textbf{Page}      & \textbf{Channels}          & \textbf{Frequency Band} & \textbf{Modulation}  & \textbf{Data Rate} \\ \hline
0                  & 11$\sim$26 (1$\sim$20)     & 2450MHz                 & O-QPSK               & 250Kb/s            \\ \hline
1                  & 1$\sim$10  (1;7$\sim$10)   & 915MHz                  & ASK                  & 250Kb/s            \\ \hline
\multirow{3}{*}{7} & \multirow{3}{*}{0$\sim$14} & \multirow{3}{*}{433MHz} & \multirow{3}{*}{MSK} & 250Kb/s            \\ \cline{5-5} 
                   &                            &                         &                      & 100Kb/s            \\ \cline{5-5} 
                   &                            &                         &                      & 31.25Kb/s          \\ \hline
9                  & 0$\sim$128 (0$\sim$26;64$\sim$128)   & 915MHz        & 2-FSK (g)            & 50Kb/s             \\ \hline
\end{tabular}
\caption{Modos de operação suportados}
\label{my-label}
\end{table}

\clearpage
\section{Procedimentos de teste}
\subsection{Controle/Referência}
\textbf{Objetivo:} Realizar as medições de RSSI Médio, LQI Médio e contagem de perda de pacotes, pacotes danificados e simetria da conexão, em condições próximas à ideal. Sendo utilizados dois nodos posocionados a cerca de 1 metro de distância.

O emissor coletará dados referentes à quantiade de pacotes enviados, quantidade de AKC's recebidos e o RSSI e o LQI médio durante a recepção dos ACK's, assim como a quantidade de erros de CRC e RX TImeouts.
\clearpage
\section{Introdução}


\end{document}
